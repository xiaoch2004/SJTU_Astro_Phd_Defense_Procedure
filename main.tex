\documentclass[a4paper, 12pt]{ctexart}
\usepackage[utf8]{inputenc}
\usepackage[colorlinks]{hyperref}
\usepackage[usenames, dvipsnames]{color}

\title{博士学位申请流程及手续}
\author{肖畅、温余彬}
\date{\today}
\newcommand{\form}[1]{\textcolor{RoyalBlue}{《#1》}}

\begin{document}

\maketitle
\tableofcontents

\section{前言}
本文是作者提交学位申请材料之后,结合\href{http://www.seiee.sjtu.edu.cn/yjb/info/13601.htm}{电院博士学位网上申请流程}\footnote{电院博士学位网上申请流程:\url{http://www.seiee.sjtu.edu.cn/yjb/info/13601.htm}}根据记忆完成的,如与实际有出入以电院网上公布流程为准。

开题前需要完成开题报告和详细的文献综述,开题答辩会需请组里老师协助寻找专家评委完成;%中期考核只需上传一张表,手续都比较简单。本文假定开题与中期考核已经完成。
%温余彬更新
中期考核(又称年度考核)需要在研究生院网站\href{https://www.gs.sjtu.edu.cn/pygl/bgxz.htm}{培养工作常用表格和文件下载}\footnote{培养工作常用表格和文件下载:\url{https://www.gs.sjtu.edu.cn/pygl/bgxz.htm}}下载名为《上海交通大学博士研究生学位论文年度考核表.doc》表格,并按照要求填写,请五位至少副高级以上的老师签字,交由学院教务处后通过后,才能操作预答辩界面界面。

\section{预答辩}
\subsection{预答辩前}
\begin{itemize}
    \item 修改好学位论文之后,在网上申请预答辩,让导师帮忙联系确定预答辩委员会成员和时间。
    \item 在网上系统中完成预答辩申请流程(需要填入论文摘要关键词等信息)。完成后找空天院教务审批,下载网上自动生成的\form{预答辩意见书}和\form{预答辩费用审批表}。在\form{预答辩意见书}上拟好意见草稿,包含自己论文的创新点与不足,可参考已毕业同学的样例。
    \item 打印至少五本学位论文,一般不需要加封皮或装订,目的仅是方便预答辩专家临场阅读。
\end{itemize}
\subsection{预答辩中}
进行预答辩会,提前拿一个打印机去会场,预答辩后现场打印\form{预答辩意见书},并请预答辩委员完成\form{预答辩意见书}和\form{预答辩费用审批表}的签字。
\subsection{预答辩后}
\begin{itemize}
    \item 如果预答辩通过,可找学科负责人(吴树范院长)完成\form{预答辩费用审批表}剩下的签字。
    \item 扫描签好字的\form{预答辩意见书}并将纸质版小心留存,正式答辩之后要交。
    \item \form{预答辩费用审批表}填上合适金额(目前\footnote{更新于\today}预答辩专家是200,答辩秘书无),交给潘志娟老师。完成后进入论文送审环节。
\end{itemize}

\section{论文送审}
%这一块印象不是很深,但基本都是电子评阅,没有纸质流程,故而比较没有什么坑,只做简略介绍。

\begin{itemize}
    \item 论文在预答辩的基础上修改好之后,即可在抽检申请界面提交盲审版(具体格式参考\href{http://www.gs.sjtu.edu.cn/info/1030/4367.htm}{研究生院学位论文评审格式要求}\footnote{研究生院学位论文评审格式要求:\url{http://www.gs.sjtu.edu.cn/info/1030/4367.htm}}),并由学院教务老师查重。
    \item 查重通过后,进行教育部盲审抽检,在此页面缴纳相应的评审费用,如果没有抽中教育部盲审,一般是一个明审专家和两个盲审专家
    \item 下载\form{通讯评议表}2份,找导师和学科负责人(吴树范院长)签字,用于答辩后找学院教务报销。
    \item 评审费用缴纳完,研究生院系统会操作盲审,明审请联系答辩秘书进行送审。
    %到陈瑞球楼328去送盲审,但不必打印纸质论文,携带签好字的\form{通讯评议审核表}即可。需要交800元盲审费用,可以微信支付。
    \item %从陈瑞球楼回来之后,暂时无事,
    等待论文评阅结果,可以在网上查到。所有结果返回并合格后,网上流程自动跳到答辩模块。
\end{itemize}

\section{正式答辩}
论文评阅结果返回后,联系老师准备正式答辩。下文提到的有些表格系统里不能直接下载的,可以到\href{http://www.gs.sjtu.edu.cn/xwxk/bgxz.htm}{研究生院网站}\footnote{研究生院网站表格下载:\url{http://www.gs.sjtu.edu.cn/xwxk/bgxz.htm}}进行下载。但需注意,有些系统里自动生成的表格是带有条码的,这些表格另外下载无效,只能用系统自动生成的。

\hypertarget{dbq}{\subsection{答辩前}}
\begin{itemize}
    \item 打印明审和盲审的\form{评阅意见书}(明审1份+盲审2份)。
    \item 按照明审和盲审意见修改论文,修改完成后在系统里下载\form{上海交通大学博士学位论文通讯评议及评阅意见反馈表},填写完成并由答辩秘书和导师签字。
    \item 联系导师确定答辩时间、地点和答辩委员会名单。
    \item 进行学术成果维护,学术成果将在\form{博士学位申请表}中自动生成。
    \item 提交答辩申请——上传答辩论文,请答辩秘书录入答辩安排,院教务审核后,可以下载三个pdf格式的表\form{答辩费用审批表}\form{答辩决议书}\form{博士学位申请表},其中只需下载并打印\form{答辩费用审批表}和\form{博士学位申请表},另外请答辩秘书从他的系统里下载word版本的\form{答辩决议书}\footnote{这是因为自己系统里的答辩决议书是pdf版本的,无法提前修改,故而不采用}。
    \item 完成\form{博士学位申请表}中各项签字,其中第6页找李灿老师填写综合评价并盖章,葛阳老师签字和自己导师。第7页指导教师意见找导师签字。
    %第7页找自己导师、学科负责人吴树范和电院3-108的教务潘飞。
    \item 打印\href{http://yjwb.seiee.sjtu.edu.cn/yjwb/info/11777.htm}{\form{上海交通大学研究生在校发表论文情况表}}\footnote{上海交通大学研究生在校发表论文情况表下载:\url{http://yjwb.seiee.sjtu.edu.cn/yjwb/info/11777.htm}}并按要求写好已发表论文的各项信息,自己和导师需要签字。同时已发表的论文打印首页,已录用未发表的论文打印全文及录用通知。
    \item 准备\form{博士学位申请表},\form{评阅意见书}(盲审+明审意见书都要),\form{通讯评议及评阅意见反馈表},\form{上海交通大学研究生在校发表论文情况表}及文章首页等,以及大论文一本(建议正式装订,答辩时候还能用),最晚答辩前一周到电院2-417找薛寒老师完成\form{博士学位申请表}第7页的学科意见以及学位评定分委员会意见。
    
    注:请务必在答辩前完成学科意见以及学位评定分委员会意见,如答辩前未完成“学位评定分委会”审批程序,则答辩无效。
    
    \item 等待电院2-417薛寒老师通知材料审核通过与否(学科意见),若通过,找电院3-108潘飞老师盖章(学位评定委员会意见)。
    
    \item 打印\form{原创性声明}和\form{论文使用授权书}并签好字、扫描。
    \item 打印\form{博士学位论文答辩表决投票},数目5-7份,等同于答辩委员会人数,每一份加盖院教务章%(我这儿有一些问题,答辩后电院老师告诉我不应盖教务章,但是后来还是让我过了,需要确定一下)
    \item 另打印空白的\form{答辩记录},然后将所有没有签字完成的表格带至答辩现场。
    \item 打印至少五本学位论文带至答辩现场。
    \item 额外做一页PPT作为答辩后拍照的背景,内容可为“XXX同学博士答辩会”与日期。
    
\end{itemize}
\hypertarget{dbz}{\subsection{答辩中}}
\begin{itemize}
    \item 答辩提问结束后,现场打印\form{答辩决议书},并请答辩主席和各个答辩委员在\form{答辩决议书}、\form{答辩费用审批表}和\form{博士学位申请表}上相应空格签字。
    \item 在\form{答辩记录}中记录好答辩委员提出的问题。理论上应由答辩秘书帮忙记录,但若答辩秘书不便记录,自己负责在答辩之后誊写到表上,并请答辩秘书签字。
    \item 收集投好票的\form{博士学位论文答辩表决投票}。
\end{itemize}

\subsection{答辩后}
    \begin{itemize}
        \item 完成论文的最后修改,扫描并复印签好字的\form{答辩决议书}。
        \item 在系统中上传归档定稿论文、签好字的\form{答辩决议书}、\form{学位论文原创性声明}、\form{学位论文使用授权书}。
        \item 上传后系统会把上述文档合成一个每页都带有条形码的归档版本论文(不需要自己事先合并)。下载之,把\form{学位论文原创性声明}和\form{学位论文使用授权书}放到其应出现的位置;\form{答辩决议书}可以附在全文最后,也可附在上述两个文档之后。
        \item 上传完成后,联系答辩秘书录入答辩结果;然后联系航空航天学院教务处审核该结果,审核后流程进入学位评审环节,自己就不能再操作了,等待学位评审结果即可。这一步完成后自己在校状态将自动变为“离校”,同时离校手续自动开启,可以上\href{http://lixiao.sjtu.edu.cn/}{离校网}\footnote{离校网:\url{http://lixiao.sjtu.edu.cn/}}查看离校手续及方法。
        \item 打印并装订带有条码的学位论文,用于学位申请。
    \end{itemize}

\section{学位材料提交}
网上流程完成后需要将纸质材料提交到系学位秘书处,控制科学与工程专业应提交到电院自动化系教务薛寒(2号楼417)老师处。提交材料清单如下:
\begin{enumerate}
    \item \form{博士学位申请表}
    \item \form{学位论文答辩决议书}(原件+复印件各1份)
    \item \form{博士学位论文评阅意见书}(盲审2份+明审1份)
    \item \form{博士学位论文通讯评议及评阅意见反馈表}
    \item \form{预答辩意见书}
    \item \form{答辩记录}
    \item \form{在校研究生发表论文情况表}(还需附上已发表论文首页+已录用未发表论文的全文及录用通知)
    \item \form{博士学位论文答辩表决投票}(5-7份)
    \item 纸质归档版装订完成的学位论文一份(带条码、签好字的答辩决议书、原创性声明和使用授权书)
\end{enumerate}

此外,还需要将如下材料给空天院教务处盖章,并交由组里的科研秘书进行报销。
\begin{enumerate}
\item \form{通讯评议审核表}

打印两份,找相应的导师和学科负责人签字后,找空天院教务处盖章,并交由组里科研秘书进行报销。

\item \form{答辩费用审批表}

打印一份,答辩专家签字后,找学科负责人,系学位委员会签字后,再找空天院教务盖章,并交由组里科研秘书进行报销
\end{enumerate}

上述所有材料都已经在\hyperlink{dbq}{答辩前}和\hyperlink{dbz}{答辩中}环节有所提及。

\section*{\hypertarget{sec:appendix}{附录1}:一些签字/盖章人员}

此名单作于2019年5月,专业:控制科学与工程。如有冲突,请以教务老师为准。
\begin{itemize}
    \item 跟答辩费用报销相关的学科负责人:吴树范
       
    包括\form{通讯评议审核表},\form{答辩费用审批表},\form{预答辩费用审批表}。
    \item (\form{答辩费用审批表})系学位评定委员会审批意见:吴树范
    %\item (答辩费用审批表)系学位评定委员会:敬忠良
    \item 报销相关经办人:潘志娟。
    
    包括\form{通讯评议审核表},\form{答辩费用审批表},\form{预答辩费用审批表}
    \item (\form{博士学位申请表}P6)院系评价签字:葛阳
    \item  (\form{博士学位申请表}P6)院系评价意见书写:李灿
    \item (\form{博士学位申请表}P7)学科(专业)意见:薛寒(电院2号楼-417)
    \item (\form{博士学位申请表}P7)学位评定分委员会:潘飞(电院3号楼-108)
    \item  空天院教务老师:何(mǐn)
    \item  自动化系教务老师:薛寒(电院2号楼-417)
    \item  电院教务老师:潘飞(电院3号楼-106)
\end{itemize}

\end{document}
